\documentclass{article}
\usepackage{amsmath,amsthm,amssymb} % Required for math stuff
\usepackage{graphicx} % Required for inserting images
\usepackage{tikz} % Required for drawing graphs
\usepackage{hyperref} % Required for hyperlinks

% Set up TikZ to add padding around pictures
\tikzset{
every picture/.style={
  execute at end picture={
    \path (current bounding box.south west) +(-0.25,-0.25) 
    (current bounding box.north east) +(0.25,0.25);
    }
  }
}

\newenvironment{theorem}[2][Theorem]{\begin{trivlist}
\item[\hskip \labelsep {\bfseries #1}\hskip \labelsep {\bfseries #2}]}
{\end{trivlist}}

\title{Project 3: Block Walking on Pascal's Triangle}
\author{Mengxiang Jiang}
\date{\today}

\begin{document}

\maketitle

\section{Introduction}
Today, I want to explore a topic I learned in high school from
a math competition book called \textit{The Art of Problem Solving Volume 2}
by Sandor Lehoczky and Richard Rusczyk \cite{lehoczky2006art}.
A familiar fact about Pascal's triangle is that its entries
correspond to the binomial coefficients, namely $\binom{n}{k}$ is the
entry in the $n$-th row and $k$-th column (starting from 0).
A related topic concerning Pascal's triangle is called block walking.
\begin{figure}[ht!]
    \label{fig:pascal}
    \centering
    \begin{tikzpicture}[
        x=1.5cm, y=1.5cm
        ]
        % Nodes
        \node (00) at ( 0,  0) {$\binom{0}{0}$};

        \node (10) at (-1, -1) {$\binom{1}{0}$};
        \node (11) at ( 1, -1) {$\binom{1}{1}$};

        \node (20) at (-2, -2) {$\binom{2}{0}$};
        \node (21) at ( 0, -2) {$\binom{2}{1}$};
        \node (22) at ( 2, -2) {$\binom{2}{2}$};

        \node (30) at (-3, -3) {$\binom{3}{0}$};
        \node (31) at (-1, -3) {$\binom{3}{1}$};
        \node (32) at ( 1, -3) {$\binom{3}{2}$};
        \node (33) at ( 3, -3) {$\binom{3}{3}$};

        % Edges
        \draw (00) -- (10);
        \draw (00) -- (11);

        \draw (10) -- (20);
        \draw (10) -- (21);

        \draw (11) -- (21);
        \draw (11) -- (22);

        \draw (20) -- (30);
        \draw (20) -- (31);

        \draw (21) -- (31);
        \draw (21) -- (32);

        \draw (22) -- (32);
        \draw (22) -- (33);
    \end{tikzpicture}
    \caption{The first few rows of Pascal's triangle}
\end{figure}
Imagine you are starting at the top of the triangle 
(see Figure \ref{fig:pascal}).
At each step, you can either move down to the left or down to the right.
For example, to get to the entry $\binom{3}{1}$, you can take the following paths:
\begin{itemize}
    \item Down left, down right, down left
    \item Down right, down left, down left
    \item Down left, down left, down right
\end{itemize}
In total, there are 3 different paths to get to $\binom{3}{1}$.
More generally, to get to the entry $\binom{n}{k}$, you need to take
$k$ steps down to the right and $n-k$ steps down to the left.
Thus, the total number of steps is $n$, and you just need to choose
which $k$ of those $n$ steps are down to the right. Therefore
this important equation holds:
\begin{equation}
    \label{eq1}
    \text{Number of paths to } \binom{n}{k} = \binom{n}{k}.
\end{equation}
We will now use this block walking idea to prove
a nice combinatorial theorem.

\section{Application of Block Walking}
\begin{theorem}{(Sum of Squares of Binomial Coefficients Identity)}
    \[
        \sum_{k=0}^{n} \binom{n}{k}^2 = \binom{2n}{n}.
    \]
\end{theorem}
\begin{proof}
    Consider a block walking scenario where you start at the top of
    Pascal's triangle and want to get to the entry $\binom{2n}{n}$.
    To get there, we simply apply equation \ref{eq1} to get that
    the total number of different paths to reach $\binom{2n}{n}$
    is $\binom{2n}{n}$.
    \\\\
    Now, let's consider an alternative way to count the number of
    paths to get to $\binom{2n}{n}$. We can break the journey into
    two parts: first, we walk down to the $n$-th row of Pascal's triangle,
    and then we walk from there down to $\binom{2n}{n}$.
    When we reach the $n$-th row, we could be at any entry $\binom{n}{k}$
    for $k=0,1,2,\ldots,n$. The number of different paths to get
    to $\binom{n}{k}$ is given by $\binom{n}{k}$ using equation \ref{eq1}.
    From $\binom{n}{k}$, we need to take another $n$ steps to reach
    $\binom{2n}{n}$. In this second part of the journey, we need
    to take $(n-k)$ steps down to the right and $k$ steps down to the left.
    The number of different paths from $\binom{n}{k}$ to $\binom{2n}{n}$
    is given by $\binom{n}{n - k}$. However, this is also equal to
    $\binom{n}{k}$ as shown below:
    \[
        \binom{n}{n - k} = \frac{n!}{(n - k)!(n - (n - k))!} = 
        \frac{n!}{(n - k)!k!} = \frac{n!}{k!(n - k)!} = \binom{n}{k}.
    \]
    Therefore, for each entry $\binom{n}{k}$ in the $n$-th row,
    the total number of paths from the top of the triangle
    to $\binom{2n}{n}$ that pass through $\binom{n}{k}$
    is given by $\binom{n}{k} \cdot \binom{n}{k} = \binom{n}{k}^2$.
    To get the total number of paths to $\binom{2n}{n}$,
    we need to sum this quantity over all possible values of $k$:
    \[
        \sum_{k=0}^{n} \binom{n}{k}^2.
    \]
    Since both counting methods count the same number of paths,
    we have
    \[
        \sum_{k=0}^{n} \binom{n}{k}^2 = \binom{2n}{n},
    \]
    which completes the proof.
\end{proof}

\section{Conclusion}
The theorem we proved isn't particularly useful on its own although it
is aesthetically pleasing in its symmetry. However, a much more 
general version of this theorem is called Vandermonde's identity.
\begin{theorem}{(Vandermonde's Identity)}
\[
    \sum_{k=0}^{r} \binom{m}{k} \binom{n}{r-k} = \binom{m+n}{r}.
\]
\end{theorem}
This identity has many applications in a huge variety of fields, and
it can also be proved using the block walking idea we discussed today,
which I encourage you to try on your own!

\begin{section}{Bibliography}
    \bibliographystyle{plain}
    \bibliography{refs}
\end{section}

\end{document}
