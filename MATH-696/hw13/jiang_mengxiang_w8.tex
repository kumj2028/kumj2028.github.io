\documentclass{article}
\usepackage{amsmath,amsthm,amssymb} % Required for math stuff
\usepackage{graphicx} % Required for inserting images


\newtheorem{theorem}{Theorem}[section]

\title{LaTeXML Proof Example}
\author{Mengxiang Jiang}
\date{\today}

\begin{document}

\maketitle

\section{Introduction}
There's an often told story of Hippasus, a mathematician during 
Pythagoras's time, that found out that the length of the diagonal of 
a square was not commensurate with the length of its side. He was 
apparently drowned for this discovery by the Pythagoreans. At some 
level, I can sympathize with the Pythagoreans, since this proof
is deeply unsettling when I first read it. Like, rational numbers
can be made as small and accurate as you want, what do you mean
there's something it can't perfectly describe? And even after
learning real analysis, the way $\sqrt{2}$ is actually described
(whether it's a Dedekind cut, equivalence class of Cauchy
sequences, or something else) leaves much to be desired.
I much prefer the algebraic treatment of $\sqrt{2}$ instead,
where just like $i$, we just claim it's not a part of the rationals
and adjoin it, without making any claims about what it really is.
So my proof won't say that $\sqrt{2}$ is irrational, but rather
that the rational numbers do not have such a number.

\section{The Proof}
\begin{theorem}
$$\forall q \in \mathbb{Q}, q^2 \ne 2.$$
\end{theorem}

\begin{proof}
Assume for contradiction that there does exist a number $q$ in the rationals whose
square is 2. We use the property that all rational numbers can be written in the
most reduced form (proof of this property left to the reader) to write 
\[
    q = \frac{a}{b},
\] 
where $a, b \in \mathbb{Z}$, $b \ne 0$, and $a$ and $b$ are coprime.
Thus
\[
    q^2 = 2 = \frac{a^2}{b^2},
\]
which implies
\[
    2b^2 = a^2.
\]
We use the property that the product of even numbers is even and the product
of odd numbers is odd (proof left to the reader) to conclude that $a$ must be
even since $a^2$ is even. Thus we can write $a=2c$ for some $c \in \mathbb{Z}$.
We then have
\[
    q^2 = 2 = \frac{(2c)^2}{b^2},
\]
which implies
\[
    2b^2 = 4c^2.
\]
Dividing both sides by 2, we have
\[
    b^2 = 2c^2,
\]
which using the latter property again tells us $b$ is even. But this
contradicts the fact that $a$ and $b$ are coprime since they both share
a factor of 2.
\end{proof}

\section{Conclusion}
This proof demonstrates that there is no rational number whose square is 2.
Here is a list of remarks regarding this proof:
\begin{itemize}
    \item This proof can be easily adapted to show that for any prime number $p$,
    there is no rational number whose square is $p$.
    \item The constructible numbers, which are numbers that can be constructed
    using a compass and straightedge, include the length of the diagonal of a
    unit square. This proof shows that the constructible numbers
    are a larger number system than the rationals.
    \item All numbers on computers are rational numbers, since they are 
    represented with a finite number of bits, which can only represent
    a finite number of fractions. This proof shows that computers
    cannot exactly represent the square root of 2, at least numerically.
\end{itemize}
Since the rationals do not contain such a number, we can adjoin $\sqrt{2}$
to them to form a larger number system that does contain such
a number, which is called the field extension
$\mathbb{Q}(\sqrt{2})$. This field extension contains all numbers of the
form $a + b\sqrt{2}$, where $a$ and $b$ are rational numbers, and is
the smallest field that contains both the rationals and $\sqrt{2}$.
In closing, we will show that $\mathbb{Q}(\sqrt{2})$ is indeed a field.
\begin{subsection}{Definition of a field}
A field is a set $F$ equipped with two binary operations, addition
and multiplication, such that the following properties hold:
\begin{itemize}
    \item Associativity of addition and multiplication: For all
    $x, y, z \in F$, we have $(x + y) + z = x + (y + z)$ and
    $(x \cdot y) \cdot z = x \cdot (y \cdot z)$.
    \item Commutativity of addition and multiplication: For all
    $x, y \in F$, we have $x + y = y + x$ and $x \cdot y = y \cdot x$.
    \item Distributivity of multiplication over addition: For all
    $x, y, z \in F$, we have $x \cdot (y + z) = x \cdot y + x \cdot z$.
    \item Additive and multiplicative identities: There exist elements
    $0, 1 \in F$ such that for all $x \in F$, $x + 0 = x$ and $x \cdot 1 = x$
    \item Additive and multiplicative inverses: For every $x \in F$,
    there exist elements $-x, x^{-1} \in F$ such that $x + (-x) = 0$ and
    $x \cdot x^{-1} = 1$, where $x^{-1}$ is defined only for $x \ne 0$.
    \item Distributivity of multiplication over addition: For all
    $x, y, z \in F$, we have $x \cdot (y + z) = x \cdot y + x \cdot z$.
\end{itemize}
\end{subsection}
Please note that many of the following steps seem trivial, but they are 
necessary since we want to only manipulate rational numbers using their 
properties without assuming they also hold for $\sqrt{2}$.
\begin{subsection}{Associativity of addition and multiplication}
Let $x = a + b\sqrt{2}$, $y = c + d\sqrt{2}$, and $z = e + f\sqrt{2}$,
where $a, b, c, d, e, f \in \mathbb{Q}$. We have
\begin{align*}
    (x + y) + z &= ((a + b\sqrt{2}) + (c + d\sqrt{2})) + (e + f\sqrt{2}) \\
    &= (a + c + (b + d)\sqrt{2}) + (e + f\sqrt{2}) \\
    &= (a + c + e) + ((b + d) + f)\sqrt{2} \\
    &= a + (c + e) + b\sqrt{2} + (d + f)\sqrt{2} \\
    &= (a + b\sqrt{2}) + ((c + d\sqrt{2}) + (e + f\sqrt{2})) \\
    &= x + (y + z).
\end{align*}
Similarly, we have
\begin{align*}
    (x \cdot y) \cdot z &= ((a + b\sqrt{2}) \cdot (c + d\sqrt{2})) \cdot (e + f\sqrt{2}) \\
    &= (ac + 2bd + (ad + bc)\sqrt{2}) \cdot (e + f\sqrt{2}) \\
    &= (ac + 2bd)e + (ac + 2bd)f\sqrt{2} + (ad + bc)e\sqrt{2} + 2(ad + bc)f \\
    &= (ace + 2bde + 2adf + 2bcf) + (acf + ade + bce)\sqrt{2} \\
    &= (a + b\sqrt{2}) \cdot ((c + d\sqrt{2}) \cdot (e + f\sqrt{2})) \\
    &= x \cdot (y \cdot z).
\end{align*}
\end{subsection}
\begin{subsection}{Commutativity of addition and multiplication}
Let $x = a + b\sqrt{2}$ and $y = c + d\sqrt{2}$,
where $a, b, c, d \in \mathbb{Q}$. We have
\begin{align*}
    x + y &= (a + b\sqrt{2}) + (c + d\sqrt{2}) \\
    &= (a + c) + (b + d)\sqrt{2} \\
    &= (c + d\sqrt{2}) + (a + b\sqrt{2}) \\
    &= y + x.
\end{align*}
Similarly, we have
\begin{align*}
    x \cdot y &= (a + b\sqrt{2}) \cdot (c + d\sqrt{2}) \\
    &= ac + 2bd + (ad + bc)\sqrt{2} \\
    &= (c + d\sqrt{2}) \cdot (a + b\sqrt{2}) \\
    &= y \cdot x.
\end{align*}
\end{subsection}
\begin{subsection}{Identities and Inverses}
The additive identity is $0 = 0 + 0\sqrt{2}$, since for any
$x = a + b\sqrt{2}$, we have
\[
    x + 0 = (a + b\sqrt{2}) + (0 + 0\sqrt{2}) = a + b\sqrt{2} = x.
\]
The multiplicative identity is $1 = 1 + 0\sqrt{2}$, since for any
$x = a + b\sqrt{2}$, we have
\[
    x \cdot 1 = (a + b\sqrt{2}) \cdot (1 + 0\sqrt{2}) = a + b\sqrt{2} = x.
\]
The additive inverse of $x = a + b\sqrt{2}$ is $-x = -a - b\sqrt{2}$, since
\[
    x + (-x) = (a + b\sqrt{2}) + (-a - b\sqrt{2}) = 0 + 0\sqrt{2} = 0.
\]
The multiplicative inverse of $x = a + b\sqrt{2}$ (for $x \ne 0$) is
\begin{align*}
    x^{-1} &= \frac{1}{a + b\sqrt{2}} \cdot \frac{a - b\sqrt{2}}{a - b\sqrt{2}} \\
    &= \frac{a - b\sqrt{2}}{a^2 - 2b^2} \\
    &= \frac{a}{a^2 - 2b^2} + \frac{-b}{a^2 - 2b^2}\sqrt{2},
\end{align*}
since
\begin{align*}
    x \cdot x^{-1} &= (a + b\sqrt{2}) \cdot \left(\frac{a}{a^2 - 2b^2} + 
    \frac{-b}{a^2 - 2b^2}\sqrt{2}\right) \\
    &= \frac{a^2 - 2b^2}{a^2 - 2b^2} + 0\sqrt{2} \\
    &= 1 + 0\sqrt{2} = 1.
\end{align*}
\end{subsection}
\begin{subsection}{Distributivity of multiplication over addition}
Let $x = a + b\sqrt{2}$, $y = c + d\sqrt{2}$, and $z = e + f\sqrt{2}$,
where $a, b, c, d, e, f \in \mathbb{Q}$. We have
\begin{align*}
    x \cdot (y + z) &= (a + b\sqrt{2}) \cdot ((c + d\sqrt{2}) + (e + f\sqrt{2})) \\
    &= (a + b\sqrt{2}) \cdot ((c + e) + (d + f)\sqrt{2}) \\
    &= a(c + e) + b(c + e)\sqrt{2} + a(d + f)\sqrt{2} + 2b(d + f) \\
    &= (ac + ae + 2bd + 2bf) + (ad + af + bc + be)\sqrt{2} \\
    &= (a + b\sqrt{2}) \cdot (c + d\sqrt{2}) + (a + b\sqrt{2}) \cdot (e + f\sqrt{2}) \\
    &= x \cdot y + x \cdot z.
\end{align*}
\end{subsection}
Thus, we have shown that $\mathbb{Q}(\sqrt{2})$ satisfies all the properties
of a field.
\end{document}
